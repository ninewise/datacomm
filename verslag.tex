\documentclass[]{article}
\usepackage[a4paper]{geometry}
\usepackage{graphicx}
%\usepackage{titling} Commented, want we gebruiken dit eigenlijk niet?
\usepackage{listings}

\usepackage[dutch]{babel}

% Wiskundige formules in LaTeX <3
\usepackage{amsmath}
\usepackage{amssymb}
\usepackage{amsfonts}

% Ik vind het nu eenmaal mooier zo.
\setlength{\parindent}{14pt}
\setlength{\parskip}{0pt}

\begin{document}

\title{Project Datacommunicatie}
\author{Enver Bral \\ Fr\'ed\'erique De Baerdemaeker \\ Ruben Taelman \\ Felix Van der Jeugt}
\date{Academiejaar 2012 - 2013}
\maketitle

\begin{section}*{Broncodering}

    \begin{subsection}*{Vraag 1}

        We verkrijgen volgende output bij het uitvoeren van
        \texttt{FrequencyCounter.m}:

        \begin{lstlisting}
            0.5443    0.0024    0.0031    0.0040
            0.0019    0.0045    0.0001    0.0026
            0.0020    0.0001    0.0044    0.0029
            0.0045    0.0025    0.0018    0.4189
        \end{lstlisting}

        Dit zijn de relatieve frequenties voor elk macroblok. De
        macroblokken zijn hier als volgt genummerd: We zetten de rijen
        van het macroblok na elkaar, en lezen als een binair getal.

        Zo wordt het macroblock \texttt{[1 0;0 1]} omgezet naar
        \texttt{1 0 0 1} en krijgt het nummer $9$.

        Met deze frequenties kunnen we de huffman-boom opstellen. Deze
        vindt u in figuur \ref{fig:manual:huffman}. Door aan de
        vertakking met het kleinste deelgewicht een nul toe te kennen,
        en een \'e\'en anders, bekomen we de codering in tabel
        \ref{tbl:manual:huffman}.

        \begin{figure}
            \centering
            \includegraphics[width=\textwidth]{manual_huffman.png}
            \label{fig:manual:huffman}
            \caption{Huffmanboom, opgesteld bij de frequenties uit
            vraag 1.}
        \end{figure}

        \begin{table}
            \centering
            \begin{tabular}{lr@{.}lr}
                \textbf{Macrosymbool} & \multicolumn{2}{l}{\textbf{Rel. freq.}} &
                \textbf{Codewoord} \\
                \hline
                \texttt{0000} & 54 & 43 \% & \texttt{1} \\
                \texttt{0001} &  0 & 24 \% & \texttt{000101} \\
                \texttt{0010} &  0 & 31 \% & \texttt{001111} \\
                \texttt{0011} &  0 & 40 \% & \texttt{00001} \\
                \texttt{0100} &  0 & 19 \% & \texttt{000000} \\
                \texttt{0101} &  0 & 45 \% & \texttt{00100} \\
                \texttt{0110} &  0 & 01 \% & \texttt{00000100} \\
                \texttt{0111} &  0 & 26 \% & \texttt{001101} \\
                \texttt{1000} &  0 & 20 \% & \texttt{000100} \\
                \texttt{1001} &  0 & 01 \% & \texttt{00000101} \\
                \texttt{1010} &  0 & 44 \% & \texttt{00011} \\
                \texttt{1011} &  0 & 29 \% & \texttt{001110} \\
                \texttt{1100} &  0 & 45 \% & \texttt{00101} \\
                \texttt{1101} &  0 & 25 \% & \texttt{001100} \\
                \texttt{1110} &  0 & 18 \% & \texttt{0000011} \\
                \texttt{1111} & 41 & 89 \% & \texttt{01} \\
            \end{tabular}
            \label{tbl:manual:huffman}
            \caption{De Huffmancode bij vraag 1.}
        \end{table}

        Nu kunnen we het gemiddeld aantal codebits per bronsymbool
        eenvoudig berekenen:

        \begin{eqnarray*}
            \mathbb{E}(n)
            &=& \frac{1}{4} \sum^{16}_{i=1}{p_i c_i} \\
            &=& \frac{1}{4*100}(1*54.43 + 6*0.24 + 6*0.31 + 5*0.40
                + 6*0.19 + 5*0.45 + 8*0.01 \\ && + 6*0.26 + 6*0.20
                + 8*0.01 + 5*0.44 + 6*0.29 + 5*0.45 + 6*0.25 \\ &&
                + 7*0.18 + 2*41.89) \\
            &=& \frac{158.77}{400} \\
            &=& 0.396925
        \end{eqnarray*}

        De afbeelding kan dus tot amper 40 procent van zijn
        oorspronkelijke grootte gereduceerd worden met behulp van
        Huffman codering.

    \end{subsection}
    
    \begin{subsection}*{Vraag 3}

        We verkrijgen volgende grafiek bij het uitvoeren van
        \texttt{vraag1\_3.m}:

        \includegraphics{vraag1_3.png}
		Hierop kunnen we de entropie zien van de macrosymbolen van grootte K. De referentielijn is hier een rechte, dit is logisch aangezien de entropie maximaal is wanneer alle symbolen met gelijke kans voorkomen.
		Vanaf K=8 komen er symbolen voor die met frequentie 0 voorkomen, dus hier is de log2 -oneindig, dus de entropie wordt hier ook oneindig groot.

    \end{subsection}

    \begin{subsection}*{Vraag 4}
	
        We verkrijgen volgende grafiek bij het uitvoeren van
        \texttt{vraag1\_4.main()}:

       \includegraphics[scale=0.75]{vraag1_4.png} \\
		Hierop kunnen we het gemiddelde aantal codebits per bronsymbool (gebruikmakend van huffmancodering) voor macrosymbolen met grootte K=1..10 en de bijhorende onder/bovengrens.
		% We nemen best 3 bits samen om het bestand optimaal te comprimeren, we zouden ook eventueel 6 kunnen nemen met als gevolg dat het coderen aanzienlijk trager verloopt.
		% We krijgen een compressiefactor van TBA
		
		\end{subsection}

      	\begin{subsection}*{Vraag 7}
        		%(zie vraag 1_7.txt)
   	\end{subsection}

\end{section}
\begin{section}*{Kanaalcodering}
\end{section}
\begin{section}*{Volledig systeem}
\end{section}
\end{document}
